\documentclass[preliminary,copyright,creativecommons,american]{eptcs}%switch a draft to preliminary or submission
\providecommand{\event}{QPL 2015} % Name of the event you are submitting to
%\usepackage{breakurl}             % Not needed if you use pdflatex only.

\usepackage{savesym}
%\usepackage{iopams}  
\savesymbol{eqalign}
\expandafter\let\csname equation*\endcsname\relax 
\expandafter\let\csname endequation*\endcsname\relax
\usepackage{nccmath}
\usepackage{amssymb}
\usepackage{amsfonts}
\usepackage{amsthm}
\newtheoremstyle{lemeq}% name
{\abovedisplayskip }% Space above
{\belowdisplayskip}% Space below
{}% Body font
{}% Indent amount
{\bfseries}% Theorem head font
{:}% Punctuation after theorem head
{\parindent}% Space after theorem head
{\thmname{#1} \thmnumber{\normalfont (#2)} \thmnote{\normalfont#3}}% Theorem head spec (can be left empty, meaning ‘normal’)
\theoremstyle{lemeq}
\newtheorem{theo}{Theorem}
%\newtheorem{thm}[theo]{Theorem}
\newtheorem{prop}[theo]{Proposition} 
\newtheorem{lemma}[equation]{Lemma}
\newtheorem{defin}{Definition}


%\usepackage{MnSymbol}
%\savesymbol{amalg}
%\usepackage{mathabx}


\usepackage[usenames,dvipsnames]{color}
\definecolor{ultramarine}{RGB}{63, 0, 255}
\definecolor{medblue}{RGB}{0, 0, 100}
\definecolor{panblue}{RGB}{0,24,150}
\definecolor{carmine}{RGB}{150, 0, 24}

\hypersetup{colorlinks,
%linkcolor=RoyalBlue,
linkcolor=medblue,
%citecolor=OrangeRed,
citecolor=carmine,
%urlcolor=BlueViolet,
urlcolor=panblue,
anchorcolor=OliveGreen}

\usepackage[square,comma,numbers,nonamebreak,sort&compress,merge]{natbib}

\usepackage{underscore}
\usepackage[utf8]{inputenc}
\usepackage[OT1]{fontenc}
\usepackage{lmodern}
%\usepackage[english]{babel}
\usepackage{microtype}

\usepackage{graphicx}
\usepackage[export]{adjustbox}
\usepackage{placeins} %for FloatBarrier

\usepackage[format=plain,font=footnotesize,skip=8pt,labelfont=bf]{caption}
\usepackage{subcaption}
\usepackage{wrapfig}

\usepackage{tabularx}
\usepackage{booktabs}
%\usepackage{tabulary}
\newcolumntype{R}{>{\raggedleft\arraybackslash}X}
\newcolumntype{C}{>{\centering\arraybackslash}X}
\newcolumntype{L}{>{\raggedright\arraybackslash}X}
\newcolumntype{J}{>{\justifying\arraybackslash}X}

\usepackage{ragged2e}%for justifying text in tables


\newcommand{\half}[1]{\nicefrac{#1}{2}}
\newcommand{\brackets}[1]{\lbrace{#1\rbrace}}

\newcommand{\ceq}[1]{Eq.\,(\ref{#1})}
\newcommand{\fig}[1]{Fig.\,\ref{#1}}
\newcommand{\tab}[1]{Table\,\ref{#1}}

\begin{document}

\title{Quick PPT conditions for GDS states}
\author{Elie Wolfe\thanks{
ORCID: \href[pdfnewwindow]{http://orcid.org/0000-0002-6960-3796}{0000-0002-6960-3796}, 
ResearchID: \href[pdfnewwindow]{http://www.researcherid.com/rid/F-7888-2014}{F-7888-2014}, 
ScopusID: \href[pdfnewwindow]{http://www.scopus.com/inward/authorDetails.url?authorID=55324585900&partnerID=MN8TOARS}{55324585900}
}
\institute{Perimeter Institute for Theoretical Physics\thanks{Research at Perimeter Institute is supported by the Government of Canada through Industry Canada and by the Province of Ontario through the Ministry of Economic Development and Innovation.}\\
Waterloo, Ontario, Canada, N2L 2Y5}
\email{\href[pdfnewwindow]{mailto:ewolfe@perimeterinstitute.ca}{ewolfe@perimeterinstitute.ca}}
%\and
%Co Author \qquad\qquad Yet S. Else
%\institute{Stanford Univeristy\\California, USA}
%\email{\quad is@gmail.com \quad\qquad somebody@else.org}
}
\def\titlerunning{GDS PPT}
\def\authorrunning{E. Wolfe}



%\subjclass[2010]{Primary: 62F15; Secondary: 62B10}

\maketitle

\begin{abstract}
We show stuff.
\end{abstract}

{\small PACS: 89.70.-a, 02.50.Tt, 02.10.Ox \hfill 2010 \textit{AMS Classification}: 62F15, 62B10}


\section{Introduction}
Various algorithms exists to study the causal structures compatible with a set of conditional independence (CI) relations \cite{pearl2009causality,spirtes2011causation,Lauritzen1990independence,studeny2005probabilistic}. A set CI relations does not aways include enough information about a probability distribution to exclude the distribution's compatibility with a given causal structure . Particularly well-studied examples include the instrumental scenario \cite{pearl2009causality}, the quantum nonlocality \cite{spekkens2012causal,pusey2014gdag,hardyinference} and the triangle scenario \cite{fritz2012bell,steudel2010ancestors}. Entropic inequalities \cite{chaves2012entropic,fritz2013marginal,chaves2014novel} are a common tool to sift through causal structures more sensitively than independence relation testing, but the entropic approach is also not fully sufficient \cite{fritz2012bell,hardyinference}.




Moreover, it is occasionally the case that a genuine causal structure, typically represented in the form of a directed acyclic graph (DAG), fails to convey certain CI relations which exist in the probability distribution  it represents. Such scenarios display ``fine tuning'' \cite{spekkens2012causal}, or are known as ``unfaithful" \cite{spirtes2011causation} or ``unstable" \cite{pearl2009causality}. An example of such a distribution is the cryptographic Vernam cipher or One-Time-Pad \cite{vernam1926cipher,singh2000code,safavi2008onetimepad}, in which a completely-random binary-string secret-key K is summed modulo two with a binary-string plaintext P in order to output the cyphertext C. In this case C is a function of, or causally dependant on, both K and P. The causal structure of such a cipher is represented by the DAG in \fig{fig:vernamstructure}. The DAG, however, only implies (unconditional) independence between P and K. However since K is uniformly random, there C and P are \emph{also} statistically independent. This is precisely the cyptographic value, namely that learning C teaches nothing about P. However this independence vanishes if one has access the the secret key. Indeed, with the secret key K one can recover P given C. Thus while the Vernam cipher probability distribution obeys both $(P\perp K\text{ })$ and $(P\perp C)$, only the first relation is inferable from the DAG using the Causal Markov condition \cite{pearl2009causality,spekkens2012causal}. The Causal Markov condition states that 
\begin{defin}
Causal Markov condition: In a directed acyclic graph, representing causal structure, every variable $X_i$ is conditionally independent with respect to any of its nondescendants given its parents, ie. $\left(X_i \perp X_j \mid \operatorname{Par}[X_i]\right), \text{ whenever }X_j\in\operatorname{NonDesc}[X_i]$.
\end{defin}
 




\nocite{*}
\bibliographystyle{apsrev4mod}
\bibliography{samplebib}
\end{document}
